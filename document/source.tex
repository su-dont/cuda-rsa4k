\documentclass[oneside,openright,12pt,final,en]{mgr}

\usepackage{hyperref}
\usepackage[T1]{fontenc}
\usepackage[utf8]{inputenc}
\usepackage[british,UKenglish,USenglish,english,american]{babel}
\usepackage{showlabels}
\usepackage{listings}
\usepackage{xcolor}
\definecolor{dkgreen}{rgb}{0,0.6,0}
\definecolor{dred}{rgb}{0.545,0,0}
\definecolor{dblue}{rgb}{0,0,0.545}
\definecolor{lgrey}{rgb}{0.9,0.9,0.9}
\definecolor{gray}{rgb}{0.4,0.4,0.4}
\definecolor{darkblue}{rgb}{0.0,0.0,0.6}
\definecolor{bluekeywords}{rgb}{0,0,1}
\definecolor{eminence}{RGB}{0,174,178}
\lstdefinelanguage{cpp}{
	backgroundcolor=\color{lgrey},  
	basicstyle=\footnotesize \ttfamily \color{black} \bfseries,   
	breakatwhitespace=false,       
	breaklines=true,               
	captionpos=b,                   
	commentstyle=\color{dkgreen},   
	deletekeywords={...},          
	escapeinside={\%*}{*)},                  
	frame=single,                  
	language=C++,                
	keywordstyle=\color{bluekeywords},  
	morekeywords={BRIEFDescriptorConfig,string,TiXmlNode,DetectorDescriptorConfigContainer,istringstream,cerr,exit}, 
	identifierstyle=\color{black},
	emph={BigInteger,cudaStream_t,dim3,DeviceWrapper},
	emphstyle={\color{eminence}},
	stringstyle=\color{blue},      
	numbers=right,                 
	numbersep=5pt,                  
	numberstyle=\tiny\color{black}, 
	rulecolor=\color{black},        
	showspaces=false,               
	showstringspaces=false,        
	showtabs=false,                
	stepnumber=1,                   
	tabsize=5,                     
	title=\lstname,                 
}
\usepackage{graphicx}
\graphicspath{ {./images/} }
\usepackage{caption}

\title{ }
\engtitle{Timing Attack Resistant Implementation of RSA on GPU.}
\author{Krzysztof Hamerski}
\supervisor{dr Maciej Gębala}
\field{Computer Science}
\specialisation{Computer Security}


\begin{document}
%\bibliographystyle{plabbrv} %tylko gdy uĹĽywamy BibTeXa, ustawia polski styl bibliografii

\maketitle 
\tableofcontents

\chapter{Introduction}

  
  
  \section{Environment Specification}
 


\chapter{Theory}

\section{RSA}

\section{Side channel attacks}

\subsection{Timing attacks}

\section{Parallel programming on CUDA}

\chapter{Implementation}

The code was written in C++ language. In order to minimize data movement between host and the device, most of logics and computation are executed fully on GPU. Implementation of RSA encryption requires only one function - modular exponentiation of a multi precision integer. Implemented Big Integer class provides much more functions, to properly handle data and validate results. Code listing below shows the header of class BigInteger.

\begin{lstlisting}[language=cpp,caption={BigInteger.h}]
class BigInteger
{
//fields
public:

// 4096 bits
static const int ARRAY_SIZE = 128;	

private:
// Magnitude array in little endian order.
// Most-significant int is mag[length-1].
// Least-significant int is mag[0].
// Allocated on the device.
unsigned int* deviceMagnitude;

// same array allocated on the host 
// provides faster access if nothing was changed
unsigned int* hostMagnitude;

// flag indicating if hostMagnitude matches deviceMagnitude
bool upToDate;

// Device wrapper instance diffrent for every integer
// to provide parallel execution
DeviceWrapper* deviceWrapper;

// methods
public:
BigInteger();
BigInteger(const BigInteger& x);
BigInteger(unsigned int value);
~BigInteger();
const unsigned int& operator[](int index);

// factory
static BigInteger* fromHexString(const char* string);
static BigInteger* createRandom(int bitLength);

// setters, getters
void set(const BigInteger& x);	
unsigned int* getDeviceMagnitude(void) const;

// arithmetics
void add(const BigInteger& x);
void subtract(const BigInteger& x);
void multiply(const BigInteger& x);
void square(void);
void mod(const BigInteger& modulus);
void multiplyMod(const BigInteger& x, const BigInteger& modulus);
void squareMod(const BigInteger& modulus);
void powerMod(BigInteger& exponent, const BigInteger& modulus);

// logics
void shiftLeft(int bits);
void shiftRight(int bits);

// extras
bool equals(const BigInteger& value) const;
int compare(const BigInteger& value) const;
int getBitwiseLengthDiffrence(const BigInteger& value) const;
int getBitwiseLength(void) const;
int getLSB(void) const;
bool testBit(int bit);
void synchronize(void);
char* toHexString(void);
void print(const char* title);

//timer
void startTimer(void);
unsigned long long stopTimer(void);

// async calls
// must call synchronize to read from
void modAsync(const BigInteger& modulus);
void multiplyModAsync(const BigInteger& x, const BigInteger& modulus);
void squareModAsync(const BigInteger& modulus);

private:

void setMagnitude(const unsigned int* magnitude);
void clear(void);
void updateDeviceMagnitiude(void);
void updateHostMagnitiude(void);
static unsigned int random32(void);

/*
Parses hex string to unsigned int type.
Accepts both upper and lower case, no "0x" at the beginning.
E.g.: 314Da43F 
*/	
static unsigned int parseUnsignedInt(const char* hexString);
};

\end{lstlisting}

Big Integer class handles mathematical logic, validates input / output, and provides comfortable and readable interface.

The intermediary between Big Integer and GPU is another class - Device Wrapper.
It handles GPU kernel launches, synchronization and data movement. Class definition is listed below.

\begin{lstlisting}[language=cpp,caption={DeviceWrapper.h}]
class DeviceWrapper
{

private:

// main stream for kernel launches
cudaStream_t mainStream;

// lauch config
dim3 block_1, block_2, block_4;
dim3 thread_warp, thread_2_warp, thread_4_warp;

// 4 ints to help store results
int* deviceWords;

// auxiliary arrays
unsigned int* device4arrays;
unsigned int* device128arrays;
unsigned int* deviceArray;

unsigned long long* deviceStartTime;
unsigned long long* deviceStopTime;

public:

DeviceWrapper();
~DeviceWrapper();

// sync
unsigned int* init(int size) const;
unsigned int* init(int size, const unsigned int* initial) const;
void updateDevice(unsigned int* device_array, const unsigned int* host_array, int size) const;
void updateHost(unsigned int* host_array, const unsigned int* device_array, int size) const;
void free(unsigned int* device_x) const;

// extras
void clearParallel(unsigned int* device_x) const;
void cloneParallel(unsigned int* device_x, const unsigned int* device_y) const;
int compareParallel(const unsigned int* device_x, const unsigned int* device_y) const;
bool equalsParallel(const unsigned int* device_x, const unsigned int* device_y) const;
int getLSB(const unsigned int* device_x) const;
int getBitLength(const unsigned int* device_x) const;
void synchronize(void);

// measure time
void startClock(void);
unsigned long long stopClock(void);

// logics
void shiftLeftParallel(unsigned int* device_x, int bits) const;
void shiftRightParallel(unsigned int* device_x, int bits) const;

// arithmetics
void addParallel(unsigned int* device_x, const unsigned int* device_y) const;
void subtractParallel(unsigned int* device_x, const unsigned int* device_y) const;
void multiplyParallel(unsigned int* device_x, const unsigned int* device_y) const;
void squareParallel(unsigned int* device_x) const;
void squareParallelAsync(unsigned int* device_x) const;
void modParallel(unsigned int* device_x, unsigned int* device_m) const;
void modParallelAsync(unsigned int* device_x, unsigned int* device_m) const;
void multiplyModParallel(unsigned int* device_x, const unsigned int* device_y, const unsigned int* device_m) const;
void multiplyModParallelAsync(unsigned int* device_x, const unsigned int* device_y, const unsigned int* device_m) const;
void squareModParallel(unsigned int* device_x, const unsigned int* device_m) const;
void squareModParallelAsync(unsigned int* device_x, const unsigned int* device_m) const;

private:
void inline addParallelWithOverflow(unsigned int* device_x, const unsigned int* device_y, int blocks) const;
};
\end{lstlisting}

Project also contains Test class to validate computations, measure times and simulate encryption.
RSA class contains single "encrypt" function, which encrypts provided value. Full class diagram is presented on figure \ref{fig:classdiagram}.

\begin{figure}
	\centering
	\includegraphics{ClassDiagram}
	\caption{Class Diagram}
	\label{fig:classdiagram}
\end{figure}


\section{Parallel equals}
\section{Parallel compare}
\section{Parallel bit length}
\section{Parallel left shift}
\section{Parallel right shift}
\section{Parallel add}
\section{Parallel subtract}
\section{Parallel multiply}
\section{Parallel modulo reduction}
\section{Parallel multiply modulo}
\section{Parallel power modulo}


\addcontentsline{toc}{chapter}{Bibliography} %utworzenie w spisie treści pozycji Bibliografia
\bibliography{bibliography} % wstawia bibliografię korzystając z pliku bibliografia.bib - dotyczy BibTeXa, jeżeli nie korzystamy z BibTeXa należy użyć otoczenia thebibliography

%opcjonalnie może się tu pojawić spis rysunków i tabel
 \listoffigures
 \listoftables
\end{document}

